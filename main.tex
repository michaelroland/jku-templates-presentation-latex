%%% Template created by Susanne Hametner and Doris Pargfrieder
%%% Template altered by Pieter-Jan Hoedt (2020)

%-=-=-=-=-=-=-=-=-=-=-=-=-=-=-=-=-=-=-=-=-=-=-=-=
%
%        LOADING DOCUMENT
%
%-=-=-=-=-=-=-=-=-=-=-=-=-=-=-=-=-=-=-=-=-=-=-=-=

% Consider options 'german' and 'handout'
% For 4:3 aspect ratio, set the option 'aspectratio=43'
\documentclass[aspectratio=169]{beamer}

% -=-=-=-=-=-=-=-=-=-=-=-=-=-=-=-=-=-=-=-=-=-=-=-=-=-=-=-=-=-=
% Choose among the following options for the JKU-beamer-theme:
% -=-=-=-=-=-=-=-=-=-=-=-=-=-=-=-=-=-=-=-=-=-=-=-=-=-=-=-=-=-=
% german
% protectframetitle
% nopagenumber
% nosectionpage
% nojkuFooter
% greyText
% RE
% SOWI
% TNF
% MED
% mac
\usetheme[nojkuFooter]{jku}

%-=-=-=-=-=-=-=-=-=-=-=-=-=-=-=-=-=-=-=-=-=-=-=-=
%        LOADING PACKAGES
%-=-=-=-=-=-=-=-=-=-=-=-=-=-=-=-=-=-=-=-=-=-=-=-=
\usepackage[utf8]{inputenc}

%-=-=-=-=-=-=-=-=-=-=-=-=-=-=-=-=-=-=-=-=-=-=-=-=
%
%	PRESENTATION INFORMATION
%
%-=-=-=-=-=-=-=-=-=-=-=-=-=-=-=-=-=-=-=-=-=-=-=-=

\title{JKU \LaTeX\ Beamer Theme}
\subtitle{A First Layout Proposal}
\author{Space for speaker names}
\institute{Space for institute information}
%\date{}
%\partnerlogo{fwf.png}

\bibliographystyle{abbrv}

\begin{document}

\jkulogoblack

\jkulogogrey

\jkulogowhite


%-=-=-=-=-=-=-=-=-=-=-=-=-=-=-=-=-=-=-=-=-=-=-=-=
%
%	TITLE PAGE
%
%-=-=-=-=-=-=-=-=-=-=-=-=-=-=-=-=-=-=-=-=-=-=-=-=

\maketitle

%-=-=-=-=-=-=-=-=-=-=-=-=-=-=-=-=-=-=-=-=-=-=-=-=
%
%	SECTION: PREREQUISITES
%
%-=-=-=-=-=-=-=-=-=-=-=-=-=-=-=-=-=-=-=-=-=-=-=-=

\section{Prerequisites}

%-=-=-=-=-=-=-=-=-=-=-=-=-=-=-=-=-=-=-=-=-=-=-=-=
%	FRAME: Installation
%-=-=-=-=-=-=-=-=-=-=-=-=-=-=-=-=-=-=-=-=-=-=-=-=

\begin{frame}[fragile]
\frametitle{Installation}

\begin{itemize}
	\item Download \texttt{jkubeamer.zip} (or \texttt{jkulatex.zip}) from \texttt{teaming.jku.at}.
	\item Extract its content (unzip) and move the files to a location on your computer where {\LaTeX} will find them (depends on your installation).
	\item Use documentclass \texttt{beamer} with your preferred options.
	\item Use JKU beamer theme: \verb+\usetheme[...]{jku}+.
	\item use ``pdflatex'' (because it includes graphics in non-PostScript-format).
\end{itemize}
\end{frame}

%-=-=-=-=-=-=-=-=-=-=-=-=-=-=-=-=-=-=-=-=-=-=-=-=
%	FRAME: Theme Package Requirements
%-=-=-=-=-=-=-=-=-=-=-=-=-=-=-=-=-=-=-=-=-=-=-=-=

\begin{frame}[containsverbatim]
\frametitle{Theme Package Requirements}

This theme requires that the following packages are installed:

\begin{columns}[t]
\begin{column}{0.5\textwidth}
\begin{itemize}
\item \lstinline!{beamer}!
\item \lstinline!{calc}!
\item \lstinline![utf8]{inputenc}!
\end{itemize}
\end{column}

\begin{column}{0.5\textwidth}
\begin{itemize}
\item \lstinline!{listings}!
\item \lstinline!{pgf}!
\item \lstinline!{xcolor}!
\end{itemize}
\end{column}
\end{columns}
\end{frame}

\subsection{JKU Theme Options}

%-=-=-=-=-=-=-=-=-=-=-=-=-=-=-=-=-=-=-=-=-=-=-=-=
%	FRAME: Theme Options
%-=-=-=-=-=-=-=-=-=-=-=-=-=-=-=-=-=-=-=-=-=-=-=-=

\begin{frame}
\frametitle{General Options}

\begin{table}[]
	\begin{tabularx}{\linewidth}{l>{\raggedright}X}
		\toprule
		\textbf{Option}			& \textbf{Description} \tabularnewline
		\midrule
		\texttt{german} & German logo, german text templates \tabularnewline
		\texttt{nosectionpage} & Section pages will be suppressed\tabularnewline
		\texttt{nopagenumber} & Page Numbers will be suppresed \tabularnewline
		\texttt{nojkuFooter} & No JKU-logo in the page footer \tabularnewline
		\texttt{greyText} & Most text-like fonts are printed darkgrey \tabularnewline
		\bottomrule
	\end{tabularx}
	\label{tab:options}
\end{table}
\end{frame}


%-=-=-=-=-=-=-=-=-=-=-=-=-=-=-=-=-=-=-=-=-=-=-=-=
%	FRAME:
%-=-=-=-=-=-=-=-=-=-=-=-=-=-=-=-=-=-=-=-=-=-=-=-=

\begin{frame}[fragile]
\frametitle{Faculty-Specific Options}

\begin{description}
 \item[RE] colors for RE-faculty
 \item[SOWI] colors for SOWI-faculty
 \item[TNF] colors for TNF-faculty
 \item[MED] colors for MED-faculty
 \item[\ldots] maybe more in the future?
\end{description}

If no faculty-specific options are given,
the package uses general JKU color styling, which is essentially b/w.
\end{frame}

%\subsubsection{JKU Theme Information}

\begin{frame}[fragile]
\frametitle{Titlepage}

A titlepage is generated with the \verb+\maketitle+-command. It uses the information given in the preamble:
 \begin{description}
  \item[Title:] title of the presentation (\verb+\title+-command)
  \item[Subtitle:] subtitle of the presentation (\verb+\subtitle+-command)
  \item[Author:] author of the presentation (\verb+\author+-command)
  \item[Institute:] institute of the autho (\verb+\institute+-command)
  \item[Date:] date of the presentation (\verb+\date+-command)
  \begin{itemize}
   \item \verb+\date{}+ suppresses the date
   \item omitting the \verb+\date+-command uses today's date
  \end{itemize}
  \item[Partnerlogo:] a logo for a partner-institution (e.g. co-sponsor) (\verb+\partnerlogo+-command). Must be a filename containing an image.

 \end{description}
\end{frame}

%-=-=-=-=-=-=-=-=-=-=-=-=-=-=-=-=-=-=-=-=-=-=-=-=
%	FRAME:
%-=-=-=-=-=-=-=-=-=-=-=-=-=-=-=-=-=-=-=-=-=-=-=-=

\begin{frame}
\frametitle{Teaser Pages}

 You might consider to put some sort of empty slide with the JKU-logo before the titlepage (like in the current presentation). This slide could be shown before the presentation begins.

 The commands for producing these slides are:
 \begin{itemize}
  \item \texttt{\textbackslash{}jkulogoblack}
  \item \texttt{\textbackslash{}jkulogogrey}
  \item \texttt{\textbackslash{}jkulogowhite}
 \end{itemize}
\end{frame}


%-=-=-=-=-=-=-=-=-=-=-=-=-=-=-=-=-=-=-=-=-=-=-=-=
%	FRAME: Table of Contents
%-=-=-=-=-=-=-=-=-=-=-=-=-=-=-=-=-=-=-=-=-=-=-=-=

\begin{frame}[containsverbatim]
\frametitle{Table of Contents}

Include a listing of the presentation's sections
For those longer presentations - keep the table of contents compact, but consider omitting an overview slide entirely, see~\cite{karol}. 
\begin{verbatim}
\begin{frame}{Table of Contents}
    \tableofcontents[hideallsubsections]
\end{frame}
\end{verbatim}
\tableofcontents[hideallsubsections]

\end{frame}

\begin{frame}[containsverbatim]
\frametitle{Table of Contents}

This is a table of contents including the subsections

\begin{verbatim}
\begin{frame}{Table of Contents}
    \tableofcontents
\end{frame}
\end{verbatim}
\tableofcontents
\end{frame}

\begin{frame}[fragile]
 \frametitle{Section Pages}

 \begin{itemize}
  \item At the beginning of each section (subsection, subsubsection) a slide will be inserted automatically telling the section title.
  \item Choose compact titles such that they fit on one line, it will look better!
  \item Section pages can be suppressed with the option \texttt{nosectionpage}.
  \item No section pages will be generated in handout-mode, see slide~\ref{handout}.
  \item Frame numbering does not count the section pages. Btw., if you want to reference a slide number, use the option \verb+[label=s]+ in the target frame. Then \verb+\ref{s}+ will give a reference to the slide number!
 \end{itemize}
 
\end{frame}

\begin{frame}[containsverbatim]
\frametitle{Items \& Enums}

\LaTeX\ does not support more than three levels, and this is good!

\begin{columns}
\begin{column}{.45\textwidth}
\begin{itemize}
	\item point 1
	\begin{itemize}
	  \item sub 1
	\begin{itemize}
	  \item subsub 1
	\end{itemize}
	\end{itemize}
\end{itemize} 
\begin{verbatim}
\begin{itemize}
	\item point 1
	\begin{itemize}
	  \item sub 1
	\begin{itemize}
	  \item subsub 1
	\end{itemize}
	\end{itemize}
\end{itemize} 
\end{verbatim}
\end{column}

\begin{column}{.45\textwidth}
\begin{enumerate}
	\item point 1
	\begin{enumerate}
	  \item sub 1
	  \begin{enumerate}
	  \item subsub 1
	\end{enumerate}
	\end{enumerate}
\end{enumerate} 
\begin{verbatim}
\begin{enumerate}
	\item point 1
	\begin{enumerate}
	  \item sub 1
	  \begin{enumerate}
	  \item subsub 1
	\end{enumerate}
	\end{enumerate}
\end{enumerate}  
\end{verbatim}
\end{column}
\end{columns}

\end{frame}

%-=-=-=-=-=-=-=-=-=-=-=-=-=-=-=-=-=-=-=-=-=-=-=-=
%	FRAME: Blocks
%-=-=-=-=-=-=-=-=-=-=-=-=-=-=-=-=-=-=-=-=-=-=-=-=

\begin{frame}[fragile]
\frametitle{Blocks}

\begin{block}{Block Title Here}
In \LaTeX-beamer it is very common to highlight content in so-called blocks. This is a standard \texttt{block}.
\end{block}

\begin{verbatim}
\begin{block}{Block Title Here}
  In \LaTeX-beamer it is very common to highlight content in 
  so-called blocks. This is a standard \texttt{block}.
\end{block} 
\end{verbatim}
\end{frame}

%-=-=-=-=-=-=-=-=-=-=-=-=-=-=-=-=-=-=-=-=-=-=-=-=
%	FRAME: Additional Blocks
%-=-=-=-=-=-=-=-=-=-=-=-=-=-=-=-=-=-=-=-=-=-=-=-=

\begin{frame}[fragile]
\frametitle{Additional Blocks}
\begin{alertblock}{Alert Block}
	This is an alert-block, if you have to use it.
\end{alertblock}

\begin{verbatim}
\begin{alertblock}{Alert Block}
  This is an alert-block, if you have to use it.
\end{alertblock}
\end{verbatim}
\end{frame}

%-=-=-=-=-=-=-=-=-=-=-=-=-=-=-=-=-=-=-=-=-=-=-=-=
%	FRAME: Environments
%-=-=-=-=-=-=-=-=-=-=-=-=-=-=-=-=-=-=-=-=-=-=-=-=

\begin{frame}
\frametitle{Environments}

\begin{definition}[Monoid]
 We call $(M,\circ)$ a \emph{monoid} if and only if
 \begin{gather*}
  \forall a,b,c\in M:\quad(a\circ b)\circ c = a\circ(b\circ c) \tag{associativity}\\
  \exists e\in M \;\forall a\in M:\quad a\circ e= e\circ a= a \tag{neutral element}
 \end{gather*}
\end{definition}

\end{frame}

\begin{frame}
\frametitle{Environments}

\begin{theorem}[Fundamental Theorem of \ldots]
 Let $(M,+)$ be a monoid. Then \ldots
\end{theorem}

\begin{proof}
 Let $M$ be an arbitrary set. We then show \ldots
\end{proof}

\begin{example}
 Put an example here.
\end{example}

\end{frame}

\begin{frame}[fragile]
\frametitle{Pre-defined Environments}

 Standard \LaTeX-beamer defines several environments like
 \begin{quote}
  theorem, corollary, fact, lemma, problem, solution, definition, definitions,
  example, and examples.
 \end{quote}

 \begin{alertblock}{German Environment Names}
  Note that if you want german environment names you have to pass the option `german' to \textbf{beamer}, it is not enough to have `german' in the JKU-beamer-theme.
 \end{alertblock}

\end{frame}

\section{Colors}

\begin{frame}
 \frametitle{Color in Presentation}

If you need colors in your presentation, use one of the pre-defined JKU-colors. 

Mac-users should use theme-option `mac'.

Consider one of the faculty-specific options, they also give decent coloring.
 
\end{frame}

\begin{frame}
\frametitle{Pre-defined JKU-Colors}

\tiny
\begin{columns}[t]

\begin{column}{0.18\textwidth}
%	Color Box: Blue
\setbeamercolor{boxjkuBlue}{bg=jkuBlue,fg=white}
\begin{beamercolorbox}[wd=0.85\linewidth,ht=5ex,dp=3ex]{boxjkuBlue}
\centering
	\texttt{jkuBlue}\\
\end{beamercolorbox}

\vspace{2em}

%	Color Box: Cyan
\setbeamercolor{boxjkuCyan}{bg=jkuCyan,fg=black}
\begin{beamercolorbox}[wd=0.85\linewidth,ht=5ex,dp=3ex]{boxjkuCyan}
\centering
	\texttt{jkuCyan}\\
\end{beamercolorbox}

\vspace{2em}

%	Color Box: Black
\setbeamercolor{boxjkuBlack}{bg=black,fg=white}
\begin{beamercolorbox}[wd=0.85\linewidth,ht=5ex,dp=3ex]{boxjkuBlack}
\centering
	\texttt{black}\\
\end{beamercolorbox}

\end{column}

\begin{column}{0.18\textwidth}
%	Color Box: Yellow
\setbeamercolor{boxjkuYellow}{bg=jkuYellow,fg=black}
\begin{beamercolorbox}[wd=0.85\linewidth,ht=5ex,dp=3ex]{boxjkuYellow}
\centering
	\texttt{jkuYellow}\\
\end{beamercolorbox}

\vspace{2em}

%	Color Box: Grey
\setbeamercolor{boxjkuGrey}{bg=jkuGrey,fg=white}
\begin{beamercolorbox}[wd=0.85\linewidth,ht=5ex,dp=3ex]{boxjkuGrey}
\centering
	\texttt{jkuGrey}\\
\end{beamercolorbox}

\vspace{2em}

%	Color Box: White
\setbeamercolor{boxjkuWhite}{bg=white,fg=black}
\begin{beamercolorbox}[wd=0.85\linewidth,ht=5ex,dp=3ex]{boxjkuWhite}
\centering
	\texttt{white}\\
\end{beamercolorbox}

\end{column}

\begin{column}{0.18\textwidth}
%	Color Box: LightGreen
\setbeamercolor{boxjkuLightGreen}{bg=jkuLightGreen,fg=black}
\begin{beamercolorbox}[wd=0.85\linewidth,ht=5ex,dp=3ex]{boxjkuLightGreen}
\centering
	\texttt{jkuLightGreen}\\
\end{beamercolorbox}

\vspace{2em}

%	Color Box: Green
\setbeamercolor{boxjkuGreen}{bg=jkuGreen,fg=black}
\begin{beamercolorbox}[wd=0.85\linewidth,ht=5ex,dp=3ex]{boxjkuGreen}
\centering
	\texttt{jkuGreen}\\
\end{beamercolorbox}

\end{column}


\begin{column}{0.18\textwidth}
%	Color Box: Purple
\setbeamercolor{boxjkuPurple}{bg=jkuPurple,fg=white}
\begin{beamercolorbox}[wd=0.85\linewidth,ht=5ex,dp=3ex]{boxjkuPurple}
\centering
	\texttt{jkuPurple}\\
\end{beamercolorbox}

\vspace{2em}

%	Color Box: Red
\setbeamercolor{boxjkuRed}{bg=jkuRed,fg=black}
\begin{beamercolorbox}[wd=0.85\linewidth,ht=5ex,dp=3ex]{boxjkuRed}
\centering
	\texttt{jkuRed}\\
\end{beamercolorbox}
\end{column}
\end{columns}
\end{frame}


\section{General Beamer Features (non JKU-Theme-specific)}

\begin{frame}
 \frametitle{General Beamer Features}

 The following tips and tricks may be known to you as a frequent beamer user.
 \begin{itemize}
  \item background coloring
  \item empty slides
  \item including pictures
  \item using columns
  \item coustomized blocks
  \item self-defined environments
  \item producing handouts
  \item 16:9 presentations, etc.
  \item \ldots
 \end{itemize}

\end{frame}

\begingroup
\setbeamercolor{background canvas}{bg=jkuLightGreen}
\begin{frame}[containsverbatim]
\frametitle{Frame Colored Backgrounds}

You can change the color of a frame background by placing the frame within a group:

\begin{verbatim}

\begingroup
\setbeamercolor{background canvas}{bg=jkuLightGreen}
\begin{frame}
	% Your frame content goes here
\end{frame}
\endgroup
\end{verbatim}

\end{frame}
\endgroup

\begingroup
\setbeamercolor{background canvas}{bg=jkuBlue}
\setbeamercolor{normal text}{fg=jkuYellow}
\usebeamercolor[fg]{normal text}
\begin{frame}[plain,containsverbatim]

Or maybe you want a blank blue frame to work with

\begin{verbatim}

\begingroup
\setbeamercolor{background canvas}{bg=jkuBlue}
\begin[plain]{frame}
	% Your frame content goes here
\end{frame}
\endgroup
\end{verbatim}

\end{frame}
\endgroup

\begingroup
\usebackgroundtemplate{\includegraphics[width=\paperwidth]{logos/jku_en_black}}
\setbeamercolor{normal text}{fg=jkuCyan}
\usebeamercolor[fg]{normal text}
\begin{frame}[plain,containsverbatim]
Or maybe you want a background picture.

The aspect-ratio of the picture should fit the presentation page. To fill the whole screen you might use \texttt{height=\ldots}, but the result might be distorted!
\begin{verbatim}

\begingroup
\usebackgroundtemplate{
  \includegraphics[width=\paperwidth]{logos/jku_en_black}}
\begin[plain]{frame}
	% Your frame content goes here
\end{frame}
\endgroup
\end{verbatim}

\end{frame}
\endgroup



%-=-=-=-=-=-=-=-=-=-=-=-=-=-=-=-=-=-=-=-=-=-=-=-=
%	FRAME: Blocks
%-=-=-=-=-=-=-=-=-=-=-=-=-=-=-=-=-=-=-=-=-=-=-=-=

\begin{frame}[fragile]
\frametitle{Custom Blocks}
\begingroup
\setbeamercolor{block title}{fg=white, bg=jkuPurple}
\setbeamercolor{block body}{bg=jkuPurple!20}
\begin{block}{Purple customization}
	Using the theme colors to generate colored blocks.
\end{block}
\endgroup
\begin{verbatim}
\begingroup
\setbeamercolor{block title}{fg=white, bg=jkuPurple}
\setbeamercolor{block body}{bg=jkuPurple!20}
\begin{block}{Custom Blocks}
    Using the theme colors to generate colored blocks.
\end{block}
\endgroup
\end{verbatim}
\end{frame}

\begin{frame}[fragile]
\frametitle{Self-defined Environments}

\newtheorem{idea}[theorem]{Idea}
\theoremstyle{definition}
\newtheorem{defi}[theorem]{My Definition}
\theoremstyle{example}
\newtheorem{ex}[theorem]{My Example}

 You can define your own environments, too.
\begin{verbatim}
 \newtheorem{idea}[theorem]{Idea}
 \theoremstyle{definition}
 \newtheorem{def}[theorem]{My Definition}
 \theoremstyle{example}
 \newtheorem{ex}[theorem]{My Example}
\end{verbatim}
 
\end{frame}

\begin{frame}[fragile]
\frametitle{Self-defined Environments}

\begin{idea}[My own idea]
 Here is a self-defined environment
\end{idea}

\begin{defi}
 Test
\end{defi}

\begin{ex}
 Test
\end{ex}
\end{frame}



%-=-=-=-=-=-=-=-=-=-=-=-=-=-=-=-=-=-=-=-=-=-=-=-=
%	FRAME: Multiple Columns
%-=-=-=-=-=-=-=-=-=-=-=-=-=-=-=-=-=-=-=-=-=-=-=-=

\begin{frame}
\frametitle{Multiple Columns}

\begin{columns}
\begin{column}{.45\textwidth}
		Lorem ipsum dolor sit amet, consectetur adipisicing elit, sed do eiusmod
		tempor incididunt ut labore et dolore magna aliqua. Ut enim ad minim veniam,
		quis nostrud exercitation ullamco laboris nisi ut aliquip ex ea commodo
		consequat. Duis aute irure dolor in reprehenderit in voluptate velit esse
		cillum dolore eu fugiat nulla pariatur.
\end{column}
\begin{column}{.45\textwidth}
		\begin{itemize}
        	\item Point 1
        	\item Point 2
		\end{itemize}
	\end{column}
	\end{columns}
\end{frame}

\begin{frame}[label=handout]
\frametitle{Producing Handouts}
  
 \begin{enumerate}
  \item You can generate a `handout'-version with the option \texttt{handout} in \alert{beamer}.
  \item Overlays will be flattened.
  \item No section pages will be generated.
  \item Note that frame-numbering will still coincide with frame-numbers in the non-handout version.
 \end{enumerate}
\end{frame}

 \begin{frame}[containsverbatim]
\frametitle{Presentations using Different Aspect Ratio}
  
 \begin{enumerate}
  \item Standard beamer supports the option \verb+aspectratio=num+, where \verb+num+ is a decimal number and
\begin{description}
 \item[43] means 4:3,
 \item[169] means 16:9, and
 \item[1610] means 16:10.
\end{description}
  \item All JKU-specific pages appear ``nice'' in all of these three formats due to adjustments in certain spacing.
  \item Standard beamer supports more than those three, but the JKU-layout is not tested against them.
 \end{enumerate}
\end{frame}

\section{Ending the Presentation}

\begin{frame}
\frametitle{Thank You Thank You Thank You Thank You Thank You Thank You Thank You Thank You Thank You Thank You}

(you see, btw., the slide title can run over several lines \ldots)

\begin{block}{Please \ldots}
  \ldots\ refrain from putting an extra slide at the end saying \alert{``Thank you for your attention''}. This is really annoying. You can say ``Thank you'' anyway, it need not be written, instead you can put a nice ``JKU page'' as the final slide! \cite{schultz,karol}
\end{block}

\end{frame}

\begin{frame}
 Further activities:
 \begin{itemize}
  \item \ldots
  \item \ldots
  \item \ldots
  \item \ldots
 \end{itemize}
 Next meeting(s):
 \begin{enumerate}
  \item \ldots
  \item \ldots
  \item \ldots
  \item \ldots
 \end{enumerate}
 
\end{frame}

\begin{jkufinalwhite}
 Further activities:
 \begin{itemize}
  \item \ldots
  \item \ldots
 \end{itemize}
%  Next meeting(s):
%  \begin{enumerate}
%   \item \ldots
%   \item \ldots
%  \end{enumerate}
 
\end{jkufinalwhite}

\jkulogogrey
\jkulogowhite

\begin{frame}
\frametitle{References}

\bibliography{references}
\end{frame}

\begin{frame}
\frametitle{JKU Theme Information}

This is an \alert{official} beamer theme for JKU University, Linz, Austria.

\vspace{1em}

\begin{alertblock}{Please help improving the style}
You are always welcome to suggest improvements, or, even better, create alternative JKU styles based on this one.
\end{alertblock}
\end{frame}

\end{document}